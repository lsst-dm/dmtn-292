
\documentclass[]{spie}

% Package imports go here.
\renewcommand{\baselinestretch}{1.0} % Change to 1.65 for double spacing

\usepackage{amsmath,amsfonts,amssymb}
\usepackage{graphicx}
\usepackage[colorlinks=true, allcolors=blue]{hyperref}
\usepackage{listings}
\usepackage{xcolor}
\usepackage{longtable}

% Local commands go here.
\newcommand{\aj}{AJ}
\newcommand{\apj}{ApJ}
\newcommand{\apjs}{ApJS}
\newcommand{\procspie}{Proc.\ SPIE}
\newcommand{\pasj}{PASJ}
% lsstdoc documentation: https://lsst-texmf.lsst.io/lsstdoc.html
\input{meta}

% Package imports go here.

% Local commands go here.

% See ASPmanual2010.pdf 2.1.4  and ManuscriptInstructions.pdf for more details
%\markboth{auth}{short title}


\newcommand{\docRef}{DMTN-292}
\newcommand{\docUpstreamLocation}{\url{https://github.com/lsst-dm/dmtn-292}}


\begin{document}
\input{authors}
\date{\today}
\title{From observatory summit to the cloud: a general approach to service deployment and configuration management}

% This can write metadata into the PDF.
% Update keywords and author information as necessary.
\hypersetup{
    pdftitle={From observatory summit to the cloud: a general approach to service deployment and configuration management},
    pdfauthor={economouf},
    pdfkeywords={}
}

\maketitle


\begin{abstract}
In order to address the challenges of the Rubin Science Platform, Rubin developed a kubernetes-based approach to service deployment with an in-house service configuration and support infrastructure called phalanx, based on ArgoCD.
It became apparent that the challenges of running a service-oriented architecture in a modern observatory summit lent themselves equally well to this approach.
In this paper we will describe how phalanx was adapted for use for telescope, instrument and sensor control services and the advantages of providing a unified service infrastructure for both control systems and data services.
\end{abstract}


\section{Introduction}

\textbf{Put your paper here.}
\vskip 0.4in

This is the Rubin Observatory overview paper: \cite{2019ApJ...873..111I}.


\appendix
% Include all the relevant bib files.
% https://lsst-texmf.lsst.io/lsstdoc.html#bibliographies
\section{References} \label{sec:bib}
\bibliographystyle{spiebib}
\bibliography{local,lsst,lsst-dm,refs_ads,refs,books}

% Make sure lsst-texmf/bin/generateAcronyms.py is in your path
\section{Acronyms} \label{sec:acronyms}
\input{acronyms.tex}
\noindent {\tiny This material or work is supported in part by the National Science Foundation through Cooperative Agreement AST-1258333 and Cooperative Support Agreement AST1836783 managed by the Association of Universities for Research in Astronomy (AURA), and the Department of Energy under Contract No. DE-AC02-76SF00515 with the SLAC National Accelerator Laboratory managed by Stanford University.
}

\end{document}
